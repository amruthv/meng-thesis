\chapter{Conclusion}
In this research paper we have examined existing techniques for motion planning. We looked at bidirectional RRTs, MCR, direct trajectories, iterative obstacle removal RRTs, and search informed iterative obstacle removal RRTs. We are interested in algorithms that are robust to planning problems with existing feasible paths and without feasible paths that require constraint removal.

Traditional motion planning techniques do not satisfy this requirement since they are not robust to unfeasible planning problems. Additionally, we have shown that MCR often underperforms in bounded iterations by taking a long time to find subpar paths as measured by covers. Amongst the algorithms in Chapter \ref{chap:algos}, we see that the direct trajectory is the worst in all scenarios. Between the IOR-RRT and search informed IOR-RRT, the search informed RRT is more likely to find collision free paths when they exist, indicating that the better strategy is to use the latter because the cost of finding paths with collisions when collision free paths exist is high in the planning process. The greedy removal strategy empirically outperforms the probabilistic removal strategy in cover score. The greedy removal strategy can also yield a higher success rate when the number of constraint removals allowed (specified by the removal frequency) is close to the number of constraint removals done with greedy removal, which is lower bounded by the true number of constraint removals needed.

The choice between the search informed IOR-RRT with greedy removal and the repeated IOR-RRT with probabilistic removal centers around computation time. We saw that in complicated unfeasible worlds with many obstacles, the repeated IOR-RRT can take almost 70\% longer to return a path at the cost of finding better covers. However, the repeated IOR-RRT only finds better covers when the MCR solution is along a non-greedy path from the start configuration to the goal configuration. In feasible worlds, which are the most common, the search informed IOR-RRT is much faster at finding a good path. Between the two options, the search informed IOR-RRT provides the best tradeoff of computation time and good path covers.

A search augmented iterative obstacle removal RRT with greedy removal is the best choice for use in a planner. While it does not always find a path with the lowest cover, it instead offers an algorithm that can find collision free paths when possible and quickly find reasonably good paths otherwise. This strategy can be used in planners that require being able to identify mechanisms for making actions feasible.