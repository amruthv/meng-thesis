\chapter{Conclusion}
In this research paper we examined existing techniques for motion planning like the RRT, the bidirectional RRT, and MCR. We propose the IOR-RRT, the repeated IOR-RRT, and the search informed IOR-RRT. We are interested in algorithms that are robust to the presence of feasible paths in planning problems.

Traditional motion planning techniques do not satisfy this requirement since they are not suited to unfeasible planning problems. Additionally, we have shown that MCR often underperforms in bounded iterations by taking a long time to find subpar paths as measured by covers. Amongst the algorithms in Chapter \ref{chap:algos}, we see that the direct trajectory is the worst in all scenarios. Between the IOR-RRT and search informed IOR-RRT, the search informed RRT is more likely to find collision free paths when they exist, suggesting that the correct choice is to use the latter because the cost of finding paths with collisions when collision free paths exist is high in the planning process. The greedy removal strategy empirically outperforms the probabilistic removal strategy in cover score. The greedy removal strategy can also yield a higher success rate when the number of constraint removals allowed (specified by the removal frequency) is close to the number of constraint removals done with greedy removal, which is lower bounded by the true number of constraint removals needed.

The choice between the search informed IOR-RRT with greedy removal and the repeated IOR-RRT with probabilistic removal centers around computation time. We saw that in complicated unfeasible worlds with many obstacles, the repeated IOR-RRT could find paths with better covers at the cost of taking almost 85\% longer to return a path. However, the repeated IOR-RRT only finds significantly better covers when the MCR solution is along a non-greedy path from the start configuration to the goal configuration. In feasible worlds, which are the most common, the search informed IOR-RRT is much faster at finding a good path. Between the two options, the search informed IOR-RRT provides the best tradeoff of computation time and good path covers.

The search informed iterative obstacle removing RRT with greedy removal is the best choice for use in a planner. While it does not always find a path with the lowest cover, it instead offers an algorithm that can find collision free paths when possible and otherwise quickly find reasonably good paths. This strategy can be used in planners that must be able to identify mechanisms for making actions feasible.